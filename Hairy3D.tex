\documentclass[10.5pt,letterpaper]{article}

\usepackage[utf8]{inputenc} % Required for including letters with accents
%\usepackage{tcolorbox} % is not requiered for the paper
\usepackage{subfig,eurosym}
\usepackage{physics}
\usepackage[svgnames]{xcolor}
\usepackage{amsmath,amssymb,amsfonts,amsthm}
\usepackage{mathtools,old-arrows,verbatim,mathrsfs,cite}
\usepackage{graphicx}
\usepackage{epstopdf,epsfig,changepage}
\newcommand\hmmax{1}
\newcommand\bmmax{1}
\usepackage{bm}
\usepackage[top=2.5cm, bottom=3cm, left=3.5cm, right=3.5cm,
heightrounded,marginparwidth=1.5cm, marginparsep=1cm]{geometry}
\usepackage[shortlabels]{enumitem}
\usepackage[square,numbers,merge,comma,sort&compress]{natbib} % bibliography
\makeatletter
\def\NAT@spacechar{\,}  % define space inside [1,\,2] ?
\makeatother
\usepackage{comment}
\usepackage{relsize,setspace,moresize}
\usepackage{latexsym,mathrsfs,calligra,aurical}
\usepackage{float,appendix,xargs,extarrows,empheq,url}

%\usepackage[breaklinks=true,backref=page]{hyperref}

\usepackage{jheppub}
\hypersetup{
	bookmarks=true,pdfmenubar=true,pdffitwindow=false,pdfpagemode={UseNone},
	pdfstartview={FitH},colorlinks=true,bookmarks=true,bookmarksnumbered=true,
	plainpages,linktoc=page,citecolor=blue,filecolor=black,
	linkcolor=Crimson,urlcolor=Green}
\usepackage{footnotebackref}
\usepackage{hypernat}


\newcommand{\tcb}{\textcolor{blue}}
\newcommand{\tcr}{\textcolor{red}}
%\renewcommand{\b}[1]{\bar{#1}}
%\renewcommand{\t}[1]{\tilde{#1}}

\DeclareFixedFont\trfont{OT1}{phv}{b}{sc}{11}
\interfootnotelinepenalty=10000

\hyphenation{sym-me-tri-za-tion}

\def\={\:=\:}
\newcommand{\N}{\mathcal{N}}
\def\pa{\partial}
\newcommand{\br}{\biggr}
\newcommand{\bl}{\biggl}
\newcommand\nv{n_\text{v}}
\newcommand\ns{n_\text{s}}
\newcommand\Ms{\mathscr{M}}
\newcommand\zb{{\bar{z}}}
\newcommand\ib{{\bar{\imath}}}
\newcommand\jb{{\bar{\jmath}}}
\newcommand\Mscal{\mathscr{M}_\text{scal}}

\renewcommand{\(}{\left(}
\renewcommand{\)}{\right)}
\renewcommand{\[}{\left[}
\renewcommand{\]}{\right]}
\def\Om{\Omega}
\newcommandx{\ETh}[2][1=M,2=\alpha,usedefault]{\Theta_{#1}{}^{#2}}
\newcommandx{\overbar}[1]{\mkern1.5mu\overline{\mkern-2.0mu#1\mkern-2.0mu}\mkern1.5mu}
\newcommandx{\overbarM}[1]{\mkern6.0mu\overline{\mkern-5.5mu#1\mkern-3.5mu}\mkern1.5mu}
\newcommandx{\overbarcal}[1]{\mkern6.0mu\overline{\mkern-5.5mu#1\mkern-1.0mu}\mkern1.5mu}
\DeclareFixedFont\trfont{OT1}{phv}{b}{sc}{11}
\DeclareMathAlphabet{\mathpzc}{OT1}{pzc}{m}{it}
\DeclareMathAlphabet{\mathcal}{OMS}{cmsy}{m}{n}
\DeclareSymbolFontAlphabet{\Scr}{rsfs}
\DeclareMathAlphabet{\mathbold}{U}{BOONDOX-ds}{m}{n}
\SetMathAlphabet{\mathbold}{bold}{U}{BOONDOX-ds}{b}{n}
\DeclareMathAlphabet{\mathcalboondox}{U}{BOONDOX-calo}{m}{n}
\SetMathAlphabet{\mathcalboondox}{bold}{U}{BOONDOX-calo}{b}{n}
\DeclareMathAlphabet{\mathbcalboondox}{U}{BOONDOX-calo}{b}{n}


\title{\centering\boldmath\LARGE\bfseries{%
		Sch-Melvin%
	}\vspace{1.25em}}


\author[a]{xxxx}
\emailAdd{xxxxx@pucv.cl}

\author[b]{David Choque,}
\emailAdd{david.choque@pucv.cl}

\author[c]{xxxx}
\emailAdd{xxxx}


\bigskip

\affiliation[a]{Universidad Nacional de San Antonio Abad del Cusco, Av. La Cultura 733, Cusco, Per\'u.}

\affiliation[b]{Pontificia Universidad Cat\'{o}lica de Valpara\'{i}so,
	Instituto de F\'{i}sica, Av.\ Brasil 2950, Valpara\'{i}so, Chile}


\bigskip


\abstract{%
	We present an exact.%
}


\date{}


\begin{document}
	
	\maketitle

%%%%
\section{Hairy-Kerr-Newman}
%%%

R: Consideramos la teor\'ia
\begin{equation}
I=\frac{1}{2\kappa}\int_{\mathcal{M}}{d^4x\sqrt{-g}\left[R-2(\pa\phi)^2-e^{-2\sqrt{3}\phi}F^2\right]}
\end{equation}
con las ecuaciones de movimiento
\begin{align}
R_{\mu\nu}-2\pa_\mu\phi2\pa_\nu\phi-2e^{-2\sqrt{3}\phi}\left(F_{\mu\alpha}F_{\nu}{}^{\alpha}-\frac{1}{4}g_{\mu\nu}F^2\right)&=0\\
\pa_\mu(\sqrt{-g}e^{-2\sqrt{3}\phi}F^{\mu\nu})&=0\\
\Box{\phi}+\frac{\sqrt{3}}{2}e^{-2\sqrt{3}\phi}F^2&=0
\end{align}
Las siguientes expresiones para los campos representan soluciones exactas [1304.5906]
\begin{align}
ds^2&=-\frac{1-Z}{H}\left[dt+\frac{2aZ\sin^2\theta d\varphi}{\sqrt{1-v^2}(1-Z)}\right]dt+H\Sigma\left(\frac{dr^2}{\Delta}+d\theta^2+\frac{K}{H\Sigma}\sin^2\theta d\varphi^2\right)\\
A&=A_tdt+A_\varphi d\varphi, \qquad \phi=-\frac{\sqrt{3}}{2}\ln{H}
\end{align}
donde
\begin{equation}
A_t=\frac{vZ}{2(1-v^2)H^2}, \quad A_\varphi=-\frac{av Z\sin^2\theta}{2\sqrt{1-v^2}}
\end{equation}
y las funciones son
\begin{align}
K&=H\Sigma+a^2H(1+ZH^{-2})\sin^2\theta\\
H&=\sqrt{\frac{1-v^2+v^2Z}{1-v^2}}, \quad Z=\frac{2mr}{\Sigma}, \\
\Sigma&=r^2+a^2\cos^2\theta, \quad \Delta=r^2-2mr+a^2
\end{align}
Estas son soluciones estacionarias, cargadas el\'ectricamente, pero sin el campo magn\'etico externo que nos interesa. Es una soluci\'on asint\'oticamente plana, $-g_{tt}=1-2M/r+\mathcal{O}(r^{-2})$, y en el l\'imite $v=0$ se reducen a las soluciones de Kerr-Newman.

\subsection{Thermodynamics}

We read the total energy from $g_{tt}$ in its asymptotic form.

\begin{equation}
M=m\left[1+\frac{v^2}{2(1-v^2)}\right]
\end{equation}
The electric charge is
\begin{equation}
Q=\frac{1}{4\pi}\oint_{s^2_\infty}{e^{-2\sqrt{3}\phi}\star F}=\frac{mv}{1-v^2}
\end{equation}


%%%%%%%%%%%%%%%%%%%%%%5
\section*{\normalsize Acknowledgments}
\vspace{-5pt}





%---- BIBLIOGRAPHY ------------------------------------------

\newpage

%\renewcommand{\refname}{...} % For modifying the bibliography heading
\hypersetup{linkcolor=blue}
\phantomsection % use it for correct TOC link !!!
\addtocontents{toc}{\protect\addvspace{4.5pt}}% add vertical space in TOC
\addcontentsline{toc}{section}{References} % add References to TOC
\bibliographystyle{JHEP}
\bibliography{bibliographyBH} % The file containing the bibliography


\end{document}